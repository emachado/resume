% Edjunior Barbosa Machado's resume
% based on: http://www.mcnabbs.org/andrew/linux/latexres/
% Created: 12 July 2013
% Last Modified: 16 Mar 2017

\documentclass[11pt,oneside]{article}
\usepackage{ae,aecompl}
\usepackage[english]{babel}
\usepackage[utf8]{inputenc}
\usepackage[T1]{fontenc}
\usepackage{geometry}

\pagestyle{empty}
\geometry{letterpaper,tmargin=1in,bmargin=1in,lmargin=1in,rmargin=1in,headheight=0in,headsep=0in,footskip=.3in}

\setlength{\parindent}{0in}
\setlength{\parskip}{0in}
\setlength{\itemsep}{0in}
\setlength{\topsep}{0in}
\setlength{\tabcolsep}{0in}

% Name and contact information
\newcommand{\name}{Edjunior Barbosa Machado}
\newcommand{\addr}{São Paulo, SP}
\newcommand{\phone}{+55 11 95331-8871}
\newcommand{\email}{edjunior@gmail.com}


%%%%%%%%%%%%%%%%%%%%%%%%%%%%%%%%%%%%%%%%%%%%%%%%%%%%%%%%%
% New commands and environments

% This defines how the name looks
\newcommand{\bigname}[1]{
	\begin{center}\selectfont\Huge\scshape#1\end{center}
}

% A ressection is a main section (<H1>Section</H1>)
\newenvironment{ressection}[1]{
	\vspace{4pt}
	{\selectfont\Large#1}
	\begin{itemize}
	\vspace{3pt}
}{
	\end{itemize}
}

% A resitem is a simple list element in a ressection (first level)
\newcommand{\resitem}[1]{
	\vspace{-4pt}
	\item \begin{flushleft} #1 \end{flushleft}
}

% A ressubitem is a simple list element in anything but a ressection (second level)
\newcommand{\ressubitem}[1]{
	\vspace{-1pt}
	\item \begin{flushleft} #1 \end{flushleft}
}

% A resbigitem is a complex list element for stuff like jobs and education:
%  Arg 1: Name of company or university
%  Arg 2: Location
%  Arg 3: Title and/or date range
\newcommand{\resbigitem}[3]{
	\vspace{-5pt}
	\item
	\textbf{#1}---#2 \\
	\textit{#3}
}

% This is a list that comes with a resbigitem
\newenvironment{ressubsec}[3]{
	\resbigitem{#1}{#2}{#3}
	\vspace{-2pt}
	\begin{itemize}
}{
	\end{itemize}
}

% This is a simple sublist
\newenvironment{reslist}[1]{
	\resitem{\textbf{#1}}
	\vspace{-5pt}
	\begin{itemize}
}{
	\end{itemize}
}



%%%%%%%%%%%%%%%%%%%%%%%%%%%%%%%%%%%%%%%%%%%%%%%%%%%%%%%%%
% Now for the actual document:

\begin{document}

% Name with horizontal rule
\bigname{\name}

\vspace{-8pt} \rule{\textwidth}{1pt}

\vspace{-1pt} {\small\itshape \addr \hfill \phone}
\vspace{-1pt} {\small\itshape \email}

\vspace{8 pt}




%%%%%%%%%%%%%%%%%%%%%%%%
\begin{ressection}{Education}

  \resbigitem{Universidade Estadual Paulista (UNESP)}{Rio Claro, SP}{Bachelor of Science (BS), Computer Science: 2001 - 2005}
	\vspace{-2pt}

\end{ressection}


%%%%%%%%%%%%%%%%%%%%%%%%
\begin{ressection}{Experience}

	\begin{ressubsec}{IBM - Linux Technology Center}{Home based}{Software Engineer - LTC POWER Toolchain: January, 2012--March, 2017}

		\ressubitem{GDB: enable and support POWER processors' hardware
		debug facilities and new instructions, as well as investigate
		and solve issues against upstream, Linux distros partners and
		IBM Advance Toolchain GDB releases.}

		\ressubitem{PAFLib: implement and document Data Stream Control
		Register support (libpaf-dsc).}

		\ressubitem{GLIBC: contribute with patches for ppc64-specific
		features (functions for shared resources hints) and general
		support (update headers with hwpoison signals).}

		\ressubitem{Continuous Integration: install and manage
		ppc64[le] virtual machines used as buildslaves for GDB.}

		\ressubitem{Support sales proof-of-concept meetings with
		prospect customers, leading performance tests, tunning and
		investigating port of Linux applications from x86 to ppc64.}

	\end{ressubsec}

	\begin{ressubsec}{IBM - Linux Technology Center}{Hortolandia, SP / Home based}{Software Engineer - Linux Defect Support: March, 2008--December, 2011}

		\ressubitem{Work as Level-3 technical support, screening,
		debugging, solving and testing issues against packages from
		Linux distros (Red Hat/Fedora, SUSE/openSUSE) in several areas
		(basesystem tools, installation and packaging, glibc, kernel).}

	\end{ressubsec}

	\begin{ressubsec}{Asga S/A}{Paulinia, SP}{Software Developer: January, 2006--February, 2008}

		\ressubitem{Design and develop embedded Linux system for
		PowerPC plataform, implementing device drivers, network
		drivers, low level applications, libraries and CGI for
		telecom devices management interfaces.}

	\end{ressubsec}

	\begin{ressubsec}{Asga S/A}{Paulinia, SP}{Intern: November, 2005--December, 2005}

		\ressubitem{Develop and test embedded Linux utilities for
		telecom devices.}

	\end{ressubsec}

	\begin{ressubsec}{Universidade Estadual de Campinas - Centro de Computação}{Campinas, SP}{Intern: Junho, 2005--Outubro, 2005}

		\ressubitem{Implement back-end support for a PHP+PostgreSQL CMS
		to provide PDF report generation.}

	\end{ressubsec}

\end{ressection}


%%%%%%%%%%%%%%%%%%%%%%%%
\begin{ressection}{Skills}

  \resitem{\textbf{Operating Systems:} Linux (Fedora/Red Hat, Debian/Ubuntu, Gentoo, ELDK)}

	\begin{reslist}{Computer Languages:}

		\ressubitem{Proficient in C, Shell Script, HTML/XML}

		\ressubitem{Familiar with Python, PowerPC Assembly, Expect, PHP, PostgreSQL, CSS, JavaScript, \LaTeX}

	\end{reslist}

	\begin{reslist}{Tools and Systems:}

		\ressubitem{Proficient in GDB/gdbserver, GCC, Autotools, Vi, Git}

		\ressubitem{Familiar with Dejagnu, KVM/Qemu, KGDB, Buildbot, svn, cvs, Docker}

	\end{reslist}


\end{ressection}


%%%%%%%%%%%%%%%%%%%%%%%%
\begin{ressection}{Achievements and Activities}

	\resitem{Promoted to IBM Staff Software Engineer (November, 2014).}

	\resitem{Presentation: ``Aprendendo a depurar programas com o GDB''.
	Presented with Sérgio Durigan Júnior at UNICAMP, Campinas - SP (July, 2011);
	PUC, Campinas - SP (September, 2012);
	UNESP, Rio Claro - SP (October, 2012).
	(source: https://github.com/emachado/gdb-unicamp2011)}

	\resitem{Presentation: ``Depurando o Kernel de forma interativa com KGDB''.
	Presented with Breno Leitão at Fórum Internacional de Software Livre 11, Porto Alegre - RS (July, 2010).
	(source: https://github.com/emachado/kgdb-fisl2010)}

\end{ressection}

\begin{ressection}{Languages}

	\resitem{Portuguese: native proficiency.}
	\resitem{English: professional working proficiency.}

\end{ressection}


\end{document}
